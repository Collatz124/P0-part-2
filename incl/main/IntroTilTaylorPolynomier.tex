\chapter{Introduktion til Taylorpolynomier}
\label{ch:tp}
% Taylorpolynomier
Taylorpolynomier er en måde at approksimere en funktions værdier omkring en værdi $a$, 
approksimationen sker ved hjælp af et $n$'te grads polynomium. Se nedenstående definition: 
\begin{defn}
    Det $n$'te grads Taylorpolynomium $P_n$ til den continuere funktion $f(x)$ omkring værdien $a$ er givet ved:
    \[
    P_n = \sum^{n}_{i=0} \frac{d^i f(a)}{dx^i} \frac{(x-a)^{i}}{i!}
    \]
    hvor $n \in \mathbb{N}$
\end{defn}
\label{def:taylorPolynomium}
Idéen er at polynomiumets $n$'te afledte i $x = a$ bliver det samme som funktionens $n$'te afledte i $x = a$, 
dette skaber en god approksimation omkring $x = a$, men garrentere ikke en god approksimation 
for alle værdier i funktionens definitations mængde: $x \in D(f(x))$. Approksimation bliver dog ofte bedre jo 
flere led Taylorpolynomiet indeholder, dette gælder dog ikke for alle funktioner, hvilket beskrives nærmere i kapitel: \refname{ch:ts}. 
En af fordelene ved at benytte et Taylorpolynomium som approksimation for en mere kompleks funktion
er at polynomier gennerelt er nemmere at integere end andre funktioner, fordi de følger reglen: $\int k x^n = \frac{k}{n + 1}x^{n + 1}$ .

% HUSK MACLAURIN POLYNOMIER

\subsection*{Eksempler på Taylorpolynomier}
