\chapter{Introduktion til Taylorpolynomier}
\label{ch:tp}
% Taylorpolynomier
Taylorpolynomier er en måde at approksimere en funktions værdier omkring en værdi $a$, 
approksimationen sker ved hjælp af et $N$'te grads polynomium. Se nedenstående definition: 
\begin{defn}
    Det $N$'te grads Taylorpolynomium $P_N$ til den kontinuere funktion $f(x)$ omkring værdien $a$ er givet ved:
    \[
    P_N = \sum^{N}_{n=0} \frac{d^n f(a)}{dx^n}\frac{(x-a)^{n}}{n!}
    \]
    hvor $n \in \mathbb{N}$
\end{defn}
\label{def:taylorPolynomium}
Ovenstående kan også skrives som:
    \[
    P_N(x)=f(a)+\frac{d^1 f(a)}{dx^1} \frac{(x-a)^{1}}{1!}+\frac{d^2 f(a)}{dx^2} \frac{(x-a)^{2}}{2!}+...+\frac{d^n f(a)}{dx^n} \frac{(x-a)^{n}}{n!}
    \]
Hvis vi kigger på den første del i ovenstående: $P_1(x)=f(a)+\frac{d^1 f(a)}{dx^1}(x-a)$ kan vi se at når differentiationen af $f(a)$ bliver ganget ind i parentesen vil vi have et førstegradspolynomium. 
Hvis vi derefter lægger den næste del til vil vi have et andengradspolynomium osv. 

Idéen er at polynomiumets $N$'te afledte i $x = a$ bliver det samme som funktionens $N$'te afledte i $x = a$, 
dette skaber en god approksimation omkring $x = a$, men garanterer ikke en god approksimation 
for alle værdier i funktionens definitions mængde: $x \in D(f(x))$. Approksimation bliver dog ofte bedre jo 
flere led Taylorpolynomiet indeholder, dette gælder dog ikke for alle funktioner, hvilket vi også så i det tidligere kapitel med $ln(x)$.
En af fordelene ved at benytte et Taylorpolynomium som approksimation for en mere kompleks funktion
er at polynomier generelt er nemmere at integrere end andre funktioner, fordi de følger reglen: $\int k x^n = \frac{k}{n + 1}x^{n + 1}$ .

% HUSK MACLAURIN POLYNOMIER

\subsection*{Eksempel på Taylor polynomium for $e^x$}
Vi har tidligere set at Taylorrækken for funktionen $f(x)=e^x$ kan skrives som:
\[
e^x=\sum^{\infty}_{n=0}\frac{x^n}{n!}
\]

I stedet for at skrive eksponentialfunktionen som en Taylorrække vil vi skrive det som et Taylorpolynomium. Det gør vi ved at tage en mængde led $N$ med  i stedet for at have uendeligt mange led med. Dette kan skrives som:
\[
P_N(x)=\sum^{\N}_{n=0}\frac{x^n}{n!}=\frac{x^0}{0!}+\frac{x^1}{1!}+\frac{x^2}{2!}+...+\frac{x^{(n-1)}}{(n-1)!}+\frac{x^n}{n!}
\]
Hvis vi vælger $N=3$ vil vi have følgende Taylor polynomium for $f(x)=e^x$:
\[
P_3=\frac{x^0}{0!}+\frac{x^1}{1!}+\frac{x^2}{2!}+\frac{x^3}{3!}=1+x+\frac{x^2}{2}+\frac{x^3}{6}
\]

Om dette tredjegradspolynomium ved vi, at i et lille interval omkring $x=0$ vil dette tredjegradspolynomium nærme sig eksponentialfunktionen. Dette kan også ses på følgende graf:

\subsection*{Eksempel på Taylor-polynomium for $sin(x)$}
Som et andet eksempel vil vi gerne lave et Taylor-polynomium for funktionen $f(x)=sin(x)$ Først findes differentialkvotienter af forskellige orden for $f(x)=sin(x)$:
\[
\frac{d}{dx}f(x)=cox(x)       
\]

\[
\frac{d^2}{dx^2}f(x)=-sin(x)
\]

\[
\frac{d^3}{dx^3}f(x)=-cos(x)
\]

\[
\frac{d^4}{dx^4}f(x)=sin(x)
\]

Dette mønster vil køre i ring så differentialkvotienten af femte orden vil være magen til differentialkvotienten af anden orden og differentialkvotienten af sjette orden vil være magen til differentialkvotienten af tredje orden osv. Vi vil gerne lave en approksimationen i $x=0$ og sætter derfor 0 ind i funktionerne:
$f(0)=0$, $\frac{d}{dx}f(0)=1$,  $\frac{d^2}{dx}f(0)=0$,                     $\frac{d^3}{dx}f(0)=-1$,    $\frac{d^4}{dx}f(0)=0$
Vi kan ud fra dette se at hvert andet led i Taylor-polynomiet vil forsvinde da det er lig med nul og hvert andet led i Taylor-polynomiet vil skiftevis blive 1 og -1. Hvis vi laver et Taylor-polynomium hvor $N=5$ vil det se sådan ud:
\[
P_5=0+1(x-0)+\frac{0}{2!}(x-0)^2-\frac{1}{3!}(x-0)^3+\frac{0}{4!}(x-0)^4\frac{1}{5!}(x-0)^5=x-\frac{x^3}{6}+\frac{x^5}{120}
\]

Om dette femtegradspolynomium ved vi, at i et lille interval omkring $x=0$ vil dette femtegradsgradspolynomium nærme sig funktionen $f(x)=sin(x)$. Dette kan også ses på følgende graf:
Generelt vil Taylorudviklingen for $f(x)=sin(x)$ følge formen:
\[
P_{2N}(x)=x-\frac{x^3}{3!}+\frac{x^5}{5!}-...+(-1)^{N-1}\frac{2N-1}{(2N-1)!}
\]


