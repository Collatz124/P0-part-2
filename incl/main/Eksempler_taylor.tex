\chapter{Taylors sætning anvendt på trigonometriske funktioner og enhedscirklen}
Dette kapitel vil vise, hvordan man anvender den før redegjort for teori på trigonometriske funktioner og enhedscirklen. Målet er at vise, hvordan man numerisk kan udregne værdier for $\sin(x)$, $\cos(x)$ og $\tan(x)$, samt hvordan man kan anvende enhedscirklen og Taylorpolynomier til at tilnærme sig $\pi$.
\section{Taylorrække for $\sin(x)$}
Når man udleder en Taylorrække er det værd at tænke på, hvilken $a$-værdi man vælger, da den ofte afgør, hvor nyttig Taylorrækken er. Målet med Taylorrækken er at gøre $\sin(x)$ nemmere at regne på, så ideelt vil man blandt $\sin(x)$ og dens afledte finde værdier for $a$, der giver heltal eller nemme brøker. Først differentieres $\sin(x)$, så man kan skabe sig et overblik over de afledte til funktionen;
\begin{align*}
f(x) &= \sin(x) \\
f'(x) &= \cos(x) \\
f''(x) &= -\sin(x) \\
f^{(3)}(x) &= -\cos(x) \\
\end{align*}
Sinusfunktionens afledte bliver ved med at svinge mellem positiv og negativ $\sin(x)$ og $\cos(x)$. Hvis man ser på enhedscirklen, så vil funktionerne alle give heltal for hver kvarte omgang, hvilket svarer værdier af $1/2\pi$ ganget med et heltal $z$, hvor $z\in\mathbb{Z}$. Hvis $a=1/2\pi z$, så giver $\sin(x)$ og dens afledte pæne værdier i form af heltallene $-1$, $0$ og $1$. Dog så vil Taylorrækken indeholde $\pi$ for alle $z \neq 0$, hvilket kan gøre udregninger senere svære. Derfor vælges $z=0$ og dermed $a=0$, hvilket giver følgende værdier for $\sin(x)$ og dens afledte;
\begin{align*}
\sin(0) &= 0 \\
\cos(0) &= 1 \\
-\sin(0) &= 0 \\
-\cos(0) &= -1 \\
\end{align*}
Taylorrækken for $\sin(x)$ omkring $0$ bliver derfor;
\begin{align*}
T_{sin} (x) &= 0+x+0 \cdot \frac{x^2}{2}-\frac{x^3}{3!}+0 \cdot \frac{x^4}{4!}+\frac{x^5}{5!}+... 
\\
T_{sin} (x) &= x-\frac{x^3}{3!}+\frac{x^5}{5!}+... \\
\end{align*}
%kilde til ulige funktioners maclaurinrækker
Siden $\sin(x)$ er en ulige funktion, da $\sin(x)=-sin(-x)$, så er $f^{n}(0)=0, \text{hvis} n \mod 2=0$. Dermed er det kun de led, hvor $x$ er opløftet i en ulige eksponent tilbage, hvor eksponenten kan udtrykkes som $2n+1$, for $n\in \mathbb{N}_0$. De resterende led skrifter mellem at være positive $1$ og negative $1$. Her ses det, at leddet $n=0$ er positivt, dermed kan man indfører $(-1)^n$ for at sikre, at leddene skifter mellem at være positive og negative i rigtig rækkefølge. Dermed bliver summen for Taylorrækken for $\sin(x)$ omkring $0$;
\begin{align*}
\sum_{n=0}^{\infty} \frac{(-1)^n}{(2n+1)!}x^{2n+1}
\end{align*}
Der er dog ingen garanti for, at rækken konvergerer med $\sin(x)$. Derfor testes kvotientkriteriet for rækken;
\begin{align*}
\lim\limits_{n \to \infty}
\left\lvert
\frac{\frac{(-1)^{n+1}}{(2(n+1)+1)!}x^{2(n+1)+1}}
{\frac{(-1)^n}{(2n+1)!}x^{2n+1}} 
\right\lvert
&=
\lim\limits_{n \to \infty}
\left\lvert
\frac{\frac{(-1)^{n+1}}{(2n+3)!}x^{2n+3}}
{\frac{(-1)^n}{(2n+1)!}x^{2n+1}}
\right\lvert 
\\
&=
\lim\limits_{n \to \infty}
\frac{\frac{\left\lvert (-1)^{n+1} \right\lvert }{(2n+3)!} \left\lvert x^{2n+3} \right\lvert }
{\frac{\left\lvert (-1)^n \right\lvert }{(2n+1)!} \left\lvert x^{2n+1} \right\lvert }
\\
&=
\lim\limits_{n \to \infty}
\frac{\frac{\left\lvert x^{2n+3} \right\lvert}{(2n+3)!}}
{\frac{\left\lvert x^{2n+1} \right\lvert}{(2n+1)!} }
\\
&=
\lim\limits_{n \to \infty}
\frac{\left\lvert x^{2n+3} \right\lvert}{(2n+3)!}
\cdot
\frac{(2n+1)!}{\left\lvert x^{2n+1} \right\lvert}
\\
&=
\lim\limits_{n \to \infty}
\frac{\left\lvert x^{2} \right\lvert}{(2n+3)(2n+2)}
=0 \\
\end{align*}
Det viser sig, at rækken konvergerer med $\sin(x)$ for alle $x$'er, da $R=\infty$, hvor $R$ er konvergensradiussen.
\begin{equation}\label{sinrække}
\sin(x)=\sum_{n=0}^{\infty} \frac{(-1)^n}{(2n+1)!}x^{2n+1}
\end{equation}
Dette er også Maclaurinrækken for $\sin(x)$, da $c=0$. Med rækken kan man danne Maclaurinpolynomier, hvor der herunder er plottet nogle Maclaurinpolynomiet for at visualisere, hvordan de approksimerer $\sin(x)$;
\begin{figure}[H]
	\centering
	\includegraphics[scale=0.4]{fig/img/taylor_sin}
	\caption{Approksimation af sinus omkring 0 af Maclaurinpolynomier af første, tredje og femte orden}
 	\label{taylor_sin}
\end{figure}
Maclaurinpolynomiet af femte orden kan med god tilnærmelse give værdier for $\sin(x)$ i intervallet $[-\pi /2; \pi /2]$, hvor den absolutte værdi af restleddet maksimalt giver $\left\lvert R_5(\pi/2) \right\lvert = \pi^6/{(6! \cdot 2^6)} = \pi^6/{46080}$. Intervallet dækker en halv svingning, og siden $sin(x)$ svinger harmonisk, så kan man approksimere værdier for $\sin(x)$, der er udenfor intervallet ved at anvende en korresponderende værdi indenfor intervallet. Eksempelvis, så kan man approksimere værdien af $\sin(t)$, hvor $t \in ]\pi/2;\pi]$, ved at udregne $p_5(\pi-t)$, da $\sin(t) = \sin(\pi-t)$. Der er dermed ikke grund til at fremstille et Maclaurinpolynomium af højere orden, da det vil gøre beregningerne sværere.
%referer til eksempel af Esben
\section{Taylorrække for $\cos(x)$}
Hvis man derefter vil udlede en Taylorrække for $\cos(x)$ omkring samme $a$-værdi, så vil man kunne differentiere Taylorrækken for $\sin(x)$, da $\frac{d}{dx}\sin(x)=\cos(x)$. Hvis man differentiere Taylorrækken i ligning \ref{sinrække}, så får man rækken;
%man kan differentiere serien p.ga. teori 19 i calculus-bogen side 536-537 section 9.5
\[
\frac{d}{dx} \sum_{n=0}^{\infty} \frac{(-1)^n}{(2n+1)!}x^{2n+1}
=
\sum_{n=0}^{\infty} \frac{(-1)^n}{(2n)!}x^{2n}
\]
Igen kan man anvende kvotientkriteriet på rækken for at se, om den konvergerer med $\cos(x)$.
\begin{align*}
\lim\limits_{n \to \infty}
\left\lvert
\frac{\frac{(-1)^{n+1}}{(2(n+1))!}x^{2(n+1)}}
{\frac{(-1)^n}{(2n)!}x^{2n}} 
\right\lvert
&=
\lim\limits_{n \to \infty}
\left\lvert
\frac{\frac{(-1)^{n+1}}{(2n+2)!}x^{2n+2}}
{\frac{(-1)^n}{(2n)!}x^{2n}}
\right\lvert 
\\
&=
\lim\limits_{n \to \infty}
\frac{\frac{\left\lvert (-1)^{n+1} \right\lvert }{(2n+2)!} \left\lvert x^{2n+2} \right\lvert }
{\frac{\left\lvert (-1)^n \right\lvert }{(2n)!} \left\lvert x^{2n} \right\lvert }
\\
&=
\lim\limits_{n \to \infty}
\frac{\frac{\left\lvert x^{2n+2} \right\lvert}{(2n+2)!}}
{\frac{\left\lvert x^{2n} \right\lvert}{(2n)!} }
\\
&=
\lim\limits_{n \to \infty}
\frac{\left\lvert x^{2n+2} \right\lvert}{(2n+2)!}
\cdot
\frac{(2n)!}{\left\lvert x^{2n} \right\lvert}
\\
&=
\lim\limits_{n \to \infty}
\frac{\left\lvert x^{2} \right\lvert}{(2n+2)(2n+1)}
=0 \\
\end{align*}
Dermed konvergerer denne række med $\cos(x)$ for alle $x$'er. Rækken er også kendt som Maclaurinrækken for $\cos(x)$;
%evt. kilde til Maclaurinserien for cosinus i bogen side 544-545 section 9.6
\begin{equation}\label{cosrække}
\cos(x)=\sum_{n=0}^{\infty} \frac{(-1)^n}{(2n)!}x^{2n}
\end{equation}
Hvorimod man kunne udlede rækken ved at arbejde med $\cos(x)$, så gjorde de kendte relationer mellem $\cos(x)$ og $\sin(x)$ det nemmere at udlede rækken med den allerede udledte række for $\sin(x)$ i ligning \ref{sinrække}. Maclaurinrækken  for $\sin(x)$ kunne man også integrerer og multiplicerer med $-1$ for at få Maclaurinrækken for $\cos(x)$, da $\int \sin(x) dx=-\cos(x)+k$, hvor $k=-\cos(0)=-1$;
\begin{align*}
-\int \sum_{n=0}^{\infty} \frac{(-1)^n}{(2n)!}x^{2n} dx
&=
-\left(\sum_{n=1}^{\infty} \frac{(-1)^n}{(2n)!}x^{2n}-1\right) \\
-\int \sum_{n=0}^{\infty} \frac{(-1)^n}{(2n)!}x^{2n} dx
&=
\sum_{n=0}^{\infty} \frac{(-1)^n}{(2n)!}x^{2n} \\
\end{align*}
Den samme række opstår, dermed kan man anvende Maclaurinrækken for $\sin(x)$ til at udlede Maclaurinrækken for $\cos(x)$ ved både differentiation og integration. Igen kan man med Maclaurinrækken danne Maclaurinpolynomier;
\begin{figure}[H]
	\centering
	\includegraphics[scale=0.4]{fig/img/taylor_cos}
	\caption{Approksimation af cosinus omkring 0 af Maclaurinpolynomier af nulte, anden og fjerde orden}
 	\label{taylor_cos}
\end{figure}
Mangen til \ref{taylor_sin}, så ses det, at Maclaurinpolynomiet af fjerde orden giver en god tilnærmelse af $\cos(x)$ i intervallet $[-\pi /2; \pi /2]$, hvor den maksimale absolutte værdi af restleddet er $\left\lvert R_4(\pi/2) \right\lvert = \pi^5/(5! \cdot 2^5) = \pi^5/3840$. $[-\pi /2; \pi /2]$ dækker et udsving af $\cos(x)$, hvilket man kan anvende til at finde værdier for $\cos(x)$ udover intervallet, igen mangen til Maclaurinpolynomiet af femte orden for $\sin(x)$.
\section{Approksimation af $\tan(x)$}
Til $\tan(x)$ kan man igen udnytte dens relation til de andre trigonometriske funktioner, hvor ligningen $\tan(x)=\frac{\sin(x)}{\cos(x)}$ er gældende. Hvis man derfor tager rækken i ligning \ref{sinrække} og dividerer med rækken i ligning \ref{cosrække}, så får man $\tan(x)$;
\begin{equation}\label{tanbrøk}
\tan(x)
=
\frac{\sum_{n=0}^{\infty} \frac{(-1)^n}{(2n+1)!}x^{2n+1}}
{\sum_{n=0}^{\infty} \frac{(-1)^n}{(2n)!}x^{2n}}
\end{equation}
Hvis man tager delsumme af de to summe med samme øvre grænse, så for man en rationel funktion, som approksimerer $\tan(x)$ omkring $0$;
\begin{figure}[H]
	\centering
	\includegraphics[scale=0.4]{fig/img/approks_tan}
	\caption{Approksimation af tangens omkring 0 via Maclaurinpolynomier af $\sin(x)$, noteret $p_n$, og $\cos(x)$, noteret $q_n$.}
 	\label{approks_tan}
\end{figure}
Approksimationen af tangens går igen tilbage til Maclaurinpolynomierne på \ref{taylor_sin}, da de rationelle funktioner på \ref{approks_tan} kan omformuleres til $R(x)=p_n(x)/p_n'(x)$. Omformuleringen giver også god mening i forhold til gennemgangen, hvor Maclaurinrækken for $\cos(x)$ blev udledt ved at differentiere Maclaurinrækken for $\sin(x)$, så $\tan(x)=\sin(x)/\frac{d}{dx}\sin(x)$. Derfor kan man med et Maclaurinpolynomium for $\sin(x)$ approksimere $\tan(x)$ omkring $0$.
\section{Approksimation af $\pi$}
Taylorpolynomier kan også anvendes, når man skal bestemme en værdi numerisk. En sådan værdi kunne være $\pi$, hvor man anvender, hvordan $\pi$ hænger sammen med andre størrelser i cirkler. En sammenhæng er arealet af en cirkel og radiussen; $A=\pi \cdot r^2$. Her kan man med fordel vælge $r=1$, så $A=\pi$ og cirklen har ligningen $x^2+y^2=1$. Ligningen kan ændres, så man får en funktion over en halvcirkel; $y=g(x)=\sqrt{1-x^2}$. Arealet under funktionen er $\pi/2$, men det er besværligt at integrere $g(x)$, derfor dannes et Taylorpolynomium over funktionen, og det integreres i stedet for $g(x)$. Til denne funktion og dens afledte findes de pæneste værdier, når $x=0$.
???differentieringen ligger muligvis i appendiks???
\begin{align*}
g(0) &= 1 \\
g'(0) &= 0 \\
g''(0) &= -1 \\
g^{(3)}(0) &= 0 \\
g^{(4)}(0) &= -3 \\
\end{align*}
Dermed kan man lave et fjerdeordens Maclaurinpolynomium.
\begin{align*}
p_{4} (x) &= 1 + 0\cdot \frac{x}{1!} - 1\cdot \frac{x^2}{2!} + 0\cdot \frac{x^3}{3!} - 3\cdot \frac{x^4}{4!}
\\
p_{4} (x) &= 1-\frac{x^2}{2}-\frac{x^4}{8} \\
\end{align*}
Maclaurinpolynomiet integreres fra $x=-1$ til $x=1$, da funktion, og dermed halvcirklen, er defineret i intervallet [-1;1].
\[
\int_{-1}^{1} 1-\frac{x^2}{2}-\frac{x^4}{8} dx = \frac{97}{60} \approx 1,6167
\]
Resultatet var en tilnærmelse af $\pi/2$, så $2\cdot 1,6167 = 3,2334$ er en tilnærmelse af $\pi$. Tilnærmelsen er dog kun med $1$ ciffer og ingen korrekte decimaler. ???programmering???



De afledte til $g(x)$, differentieret med kædereglen og produktreglen;
\begin{align*}
g(x) &= \sqrt{1-x^2} \\
\end{align*}
\begin{align*}
g'(x) &= \frac{1}{2\sqrt{1-x^2}} (-2x)\\
&= \frac{-x}{\sqrt{1-x^2}} \\
\end{align*}
\begin{align*}
g''(x) &= \frac{-1}{\sqrt{1-x^2}} -x\left(\frac{-1}{2}\right)(-2x)\frac{1}{\sqrt{1-x^2}(1-x^2)}\\
&= \frac{-1}{\sqrt{1-x^2}} -\frac{x^2}{\sqrt{1-x^2}(1-x^2)}\\
&= \frac{-1}{\sqrt{1-x^2}}\left(\frac{x^2}{1-x^2}+1\right)\\
\end{align*}
\begin{align*}
g^{(3)}(x) &= \frac{-x}{\sqrt{1-x^2}(1-x^2)}\left(\frac{x^2}{1-x^2}+1\right)- \frac{1}{\sqrt{1-x^2}}\left(\frac{2x}{1-x^2}+\frac{2x^3}{(1-x^2)^2}\right)\\
&= \frac{-1}{\sqrt{1-x^2}}\left(\frac{x^3}{(1-x^2)^2}+\frac{x}{1-x^2}\right)- \frac{1}{\sqrt{1-x^2}}\left(\frac{2x}{1-x^2}+\frac{2x^3}{(1-x^2)^2}\right)\\
&=\frac{-1}{\sqrt{1-x^2}}\left(\frac{3x^3}{(1-x^2)^2}+\frac{3x}{1-x^2}\right) \\
\end{align*}
\begin{align*}
g^{(4)}(x) &= \frac{-x}{\sqrt{1-x^2}(1-x^2)}\left(\frac{3x^3}{(1-x^2)^2}+\frac{3x}{1-x^2}\right)-\frac{1}{\sqrt{1-x^2}}\left(\frac{9x^2}{(1-x^2)^2}+\frac{12x^4}{(1-x^2)^3}+\frac{3}{1-x^2}+\frac{6x^2}{(1-x^2)^2}\right) \\
&= \frac{-1}{\sqrt{1-x^2}}\left(\frac{3x^4}{(1-x^2)^3}+\frac{3x^2}{(1-x^2)^2}\right)-\frac{1}{\sqrt{1-x^2}}\left(\frac{9x^2}{(1-x^2)^2}+\frac{12x^4}{(1-x^2)^3}+\frac{3}{1-x^2}+\frac{6x^2}{(1-x^2)^2}\right) \\
&= \frac{-1}{\sqrt{1-x^2}} \left(\frac{15x^4}{(1-x^2)^3}+\frac{18x^2}{(1-x^2)^2}+\frac{3}{1-x^2}\right)\\
\end{align*}