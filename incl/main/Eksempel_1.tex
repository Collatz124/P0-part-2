\section{Eksempel}
Som et eksempel vil vi finde restleddet i approksimationen for eksponentialfunktionen.
Vi har tidligere set at Taylorpolynomiet for eksponentialfunktionen approksimeret i $x=0$ kan beskrives således:

\[
P_N(x)=\sum^{N}_{n=0}\frac{x^{0}}{0!}+\frac{x^{1}}{1!}+..+\frac{x^{n-1}}{(n-1)!}\frac{x^{n}}{n!}
\]

Vi vælger at kigge på Taylorpolynomiet hvor $N=2$:

\[
P_2(x)=\frac{x^{0}}{0!}+\frac{x^{1}}{1!}+\frac{x^{2}}{2!}=1+x+\frac{x^{2}}{2}
\]

Vi vælger x-værdien til at være $1$ og approksimerer:
\[
e^{1}=f(1)\approx P_{2}(1)=1+1+\frac{1^{2}}{2}=\frac{5}{2}=2.5
\]

\[
e^{1}=2.718\approx 2.5
\]

Nu kan vi finde restleddet for approksimationen.
Vi ved at $\frac{d^3}{dx^3}f(x)=e^{x}$ derfor ved vi også at $\frac{d^3}{dx^3}f(s)=e^{s}$ hvor $s$ er et tal mellem 0 og 1. Om $s$-værdien kan vi altså sige at $0<s<1$ da $s<1$ må $e^{s}<e$ da $\frac{d^3}{dx^3}f(s)=e^{s}$ må $\frac{d^3}{dx^3}f(s)<e$.

Restleddet for approksimationen af $e^{x}\approx P_n(x)$ kan beskrives som:
\[
R_N(x)={\frac{e^{s}}{(n+1)!}}\cdot x^{n+1}
\]

\[
R_2(1)<\frac{e}{(2+1)!}\cdot1^{(2+1)}=\frac{e}{6}
\]

Vi ved nu at $f(1)$ ligger i intervallet $(2.5-\frac{e}{6} , 2.5+\frac{e}{6})$. Vi kan ikke sige hvad den eksakte værdi af restleddet er, da vi ikke kender den eksakte værdi af $s$.
