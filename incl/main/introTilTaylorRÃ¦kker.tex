\chapter{Introduktion til Taylorrækker}
\label{ch:tr}
En Taylorrække er en type potensrække, som kan bruges til at approksimere funktioner.
\begin{defn}
    \textbf{Taylorrækker}\\
    En Taylorrække $T_f(x)$ for den kontinuere funktion $f(x)$ omkring værdien $a$ er givet ved den uendelige sum:
    \[
        T_f(x) = \sum^{\infty}_{n=0} \frac{f^{(n)}(a)}{n!} (x-a)^{n}
        = f(a) + \frac{f'(a)}{1!}(x-a)^{1} + \frac{f''(a)}{2!}(x-a)^{2} + \cdots
    \]
    Hvis $a = 0$ kaldes Taylorrækken i stedet for en Maclaurinrække.
\end{defn}
\section{Den grundlæggende idé med Taylorrækker}
\label{def:taylorrække} Idéen med Taylorrækker er at de afledte for rækken og for funktionen er den samme i punktet $a$, 
og at dette vil skabe en god approksimation omkring punktet $a$. 
\begin{thm}
    Taylorrækkens $n$'te afledte har samme værdi som funktionens $n$'te afledte i punktet $a$, 
    det vil sige: $T_f^{(n)}(a) = f^{(n)} (a)$
\end{thm}
\begin{proof}
    % Viser det i den uendelige sum
    lad Taylorrækken $T_f(x)$ være givet for funktionen $f(x)$: 
    \[
        T_f(x) = \sum^{\infty}_{n=0} \frac{f^{(n)}(a)}{n!} (x-a)^{n}
        = f(a) + \frac{f'(a)}{1!}(x-a)^{1} + \frac{f''(a)}{2!}(x-a)^{2} + \cdots
    \]
    Når rækken differenceres i forhold til x, forsvinder det første led og de resterende led ændres i mønsteret: $\frac{d}{dx} \frac{f^{(n)}(a)}{n!} (x-a)^{n} = \frac{f^{(n)}(a)}{(n-1)!} (x-a)^{(n - 1)}$ % KILDE TIL BOGEN (SOM SIGER AT EN UENDELIG RÆKKE KAN DIFFERENCERES)
    \[
        T_f'(x) = \frac{f'(a)}{0!}(x-a)^0 + \frac{f''(a)}{1!}(x-a)^{1} + \cdots = f'(a) + \frac{f''(a)}{1!}(x-a)^{1} + \frac{f'''(a)}{2!}(x-a)^{2} + \cdots
    \]
    Når processen gentages $n$ gange opnåes følgende udtryk for $T_f^{(n)}(x)$:
    \[
        T_f^{(n)}(x) = f^{(n)}(a) + \frac{f^{(n + 1)}(a)}{1!}(x-a)^{1} + \frac{f^{n + 2}(a)}{2!}(x-a)^{2} + \cdots
    \]
    Da de afledte skal være ens i punktet $a$ indsættes $x = a$:
    \begin{align*}
        T_f^{(n)}(a) &= f^{(n)}(a) + \frac{f^{(n + 1)}(a)}{1!}(a-a)^{1} + \frac{f^{(n + 2)}(a)}{2!}(a-a)^{2} + \cdots \\
                   &= f^{(n)}(a) + \frac{f^{(n + 1)}(a)}{1!}(0)^{1} + \frac{f^{(n + 2)}(a)}{2!}(0)^{2} + \cdots \\
                   &= f^{(n)}(a)        
    \end{align*}
    Derved må $T_f^{(n)}(a) = f^{(n)}(a)$
\end{proof}
\section{Radius for konvergens}
Selvom en Taylorrække normalt approksimerer en funktion relativt præcist, kan det ske at
approksimationen kun er god indenfor et interval af definitionsmængden $D(f)$. Dette beskrives bl.a. ved hjælp af rækkens
radius for konvergens:
\begin{defn} % Definitionen på radius for konvergens
    \textbf{Radius for konvergens}\\
    Med en Taylorrækkes radius for konvergens $r$ menes, der for hvilken del af $D(f)$ følgende gælder:
    \[
        T_f(x) = f(x)
    \]
    området opskrives ofte ved $|x-a| < r$.
\end{defn}
\label{def:radiusForKonvergens}
Ved hjælp af $R$ kan Taylorrækkens konvergens klassificeres indenfor 3 forskellige klasser.
\begin{enumerate} % klassification af rækkens radius for konvergens.
    \item $r = 0$ i så fald gælder $\lim_{n \rightarrow \infty} T_f(x) = f(x)$ kun i punktet $x = a$,
    hvilket vil sige at Taylorrækken slet ikke konvergerer ud over punktet $x = a$
    \item $r = \infty$ i så fald gælder $T_f(x) = f(x)$ for alle $x \in \mathbb{R}$,
    hvilket vil sige at Taylorrækken $T_f$ konvergerer imod funktionen $f(x)$ for alle værdier.
    \item $r = c$ hvor $c \in \mathbb{R}\setminus\{0\}$ i så fald konvergere Taylorrækken kun i området $|x-a| < r$.
\end{enumerate}
For at finde konvergensradiusen for en Taylorrække kan koefficientkriteriet benyttes, ligesom det kan ved andre potensrækker.
\begin{defn} % Definitation af koefficient kriteriet TODO: Kig ret så det bruger L
    \textbf{koefficientkriteriet}\\For en potensrække kan koefficient kriteriet opskrives som:
    \[
        L = \lim_{n \rightarrow \infty} \left\lvert \frac{c_{n + 1}}{c_n} \right\lvert
    \]
    Hvor $c_n$ for Taylorrækker er givet ved $c_n = \dfrac{f^{(n)}(a)}{n!}(x-a)^n$.
    \begin{enumerate}
        \item hvis $L = 1$ er kan der ikke konkluderes noget udfra testen
        \item hvis $L < 1$ konvergere rækken, i så fald er konvergensradiusen givet ved $R = \dfrac{1}{L}$
        \item hvis $L > 1$ divergere rækken, i så fald gælder $R = 0$
    \end{enumerate}
\end{defn} % https://blogs.ubc.ca/infiniteseriesmodule/appendices/proof-of-the-ratio-test/proof-of-the-ratio-test/
\label{def:koefficientKriteriet}
koefficient kriteriet er bevist i \citep{ratiotest}
\section{Eksempler}
\subsection*{\textbf{Eksempel:} Maclaurinrækken for $e^x$ og dens radius for konvergens} 
Der ønskes en Maclaurinrække for beregning af funktionsværdien for funktionen $f(x) = e^x$.
Da $\dfrac{d}{dx}e^x = e^x$ kan Maclaurinrækken opskrives som:
\[
    T_{e^x}(x) = \sum^\infty_{n = 0} \frac{e^0}{n!}x^n = \sum^\infty_{n = 0} \frac{x^n}{n!}
\]
Herefter benyttes koefficient kriteriet til at beregne rækkens radius for konvergens:
\[
    L = \lim_{n \rightarrow \infty} \left\lvert \frac{\frac{x^{(n+1)}}{(n+1)!}}{\frac{x^n}{n!}} \right\lvert
\]
Der ses bort fra den absolutte værdi fordi $(n + 1)! > n! > 0$, $x^{(n+1)} > 0$ og $x^{n} > 0$:
\[
    L = \lim_{n \rightarrow \infty} \frac{\frac{x^{(n+1)}}{(n+1)!}}{\frac{x^n}{n!}} = \lim_{n \rightarrow \infty} \frac{x^{(n+1)} \cdot \frac{1}{(n+1)!}}{x^n \cdot \frac{1}{n!}} = x \lim_{n \rightarrow \infty} \frac{n!}{(n+1)!} = x \lim_{n \rightarrow \infty} \frac{1}{n} = 0
\]
Da $L < 1$ må radiusen for konvergens være givet ved: $R = \dfrac{1}{L} = \dfrac{1}{0} \Leftrightarrow R = \infty$
rækken konvergere altså for alle $x \in \mathbb{R}$. Derved kan følgende udtryk opstilles:
\[
    e^x = \sum^\infty_{n = 0} \frac{x^n}{n!}  
\]
\subsection*{\textbf{Eksempel:} Taylorrækken for $\ln(x)$ omkring $a = 1$ og dens radius for konvergens} % Taylorrække for ln(x)
Taylorrækken for $\ln(x)$ er givet ved: % TODO: Ret til ny brug af koefficient kriteriet
\[
    T_{\ln}(x) = \ln(a) + \frac{\ln'(a)}{1!}(x-a)^1 + \frac{\ln''(a)}{2!}(x-a)^2 + \cdots
\]
Hvilket kan opskrives pænere som:
\[
    T_{ln}(x) = ln(a) + \sum^{\infty}_{n = 1} \frac{(-1)^{(n+1)}}{n!} \cdot \frac{(n-1)!}{a^n} (x-a)^{n}
\]
$a = 1$ indsættes:
\[
    T_{ln}(x) = \sum^{\infty}_{n = 1} \frac{(-1)^{(n+1)}}{n!} \cdot \frac{(n-1)!}{1} (x-1)^{n}
    = \sum^{\infty}_{n = 1} \frac{(-1)^{(n+1)}(x-1)^{n}}{n}
\]
Nu beregnes Taylorrækkens radius for konvergens: 
\[
    \frac{1}{R} = \lim_{n \rightarrow \infty} \left\lvert \frac{\frac{(-1)^{(n+2)}}{n+1}}{\frac{(-1)^{(n+1)}}{n}} \right\lvert
\]
Da både $\left\lvert -1^{(n+2)} \right\lvert = 1$ og $\left\lvert-1^{(n + 1)} \right\lvert = 1$ må gælde kan udtrykket forsimples yderligere da den absolute værdi kan ophæves
\[
    \frac{1}{R} = \lim_{n \rightarrow \infty} \frac{\frac{1}{n+1}}{\frac{1}{n}} = \lim_{n \rightarrow \infty} \frac{n}{n+1} = 1
\]
$\frac{1}{R} = 1 \Leftrightarrow R = 1$ må gælde fordi både tæller og nævner går imod uendelig. 
derved kan Taylorrækkens konvergens område opskrives som $a-R < x < a+R \Leftrightarrow 0 < x < 2$ 
derved kan følgende udtryk opstilles: $ln(x) = T_{ln}(x)$ hvor $ 0 < x < 2$