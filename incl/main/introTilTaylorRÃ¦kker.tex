\chapter{Introduktion til Taylorrækker} % TODO: erstat $d^nf(a) / dx^n$ med $f^n(a)$
\label{ch:tr}
Hvis Taylorpolynomiet indeholder udendeligt mange led, 
kalder man det for en Taylorrække i stedet for et polynomiem. 
\begin{defn}
    En Taylorrække $T_f(x)$ for den kontinuere funktion $f(x)$ omkring værdien $a$ er givet ved den uendelige sum:
    \[
        T_f(x) = \sum^{\infty}_{n=0} f^n(a) \frac{(x-a)^{n}}{n!} = f(a) + f'(a) \frac{(x-a)^{1}}{1!} + 
        f''(a) \frac{(x-a)^{2}}{2!} + \ldots
    \]
    Hvis $a = 0$ kaldes Taylorrækken i stedet for en Maclaurin række.
\end{defn}
\label{def:taylorrække}
Idéen med at de afledte af Taylorrækken og for funktionen er de samme i punktet $a$ 
ses igen her ligesom ved Taylorpolynomier beskrevet i kapitel \refname{ch:tp}
Her er dog uendeligt mange led, hvilket skaber en bedre approksimation omkring værdien $a$. 
Som det kort blev nævnt i kaptiel \refname{ch:tp} er det dog ikke altid at en taylor approksimation bliver bedre
når der indgår flere led, i denne sammenhæng er rækkens radiusen for konvergens en vigtig egenskab,
fordi radiusen beskriver i hvilket område omkring $a$ approksimation bliver bedre, når flere led indgår.
\begin{defn} % Definitionen på radius for konvergens
    Med en Taylorrækkes radius for konvergens $R$ menes, der for hvilket del af $x \in \mathbb{R}$ gælder:
    \[
        \lim_{n \rightarrow \infty} T_f(x) = f(x)
    \]
    området opskrives ofte ved $|x-a| < R$.
\end{defn}
Ved hjælp af $R$ kan taylorrækkens konvergens klassificeres indenfor 3 forskellige klasser.
\begin{enumerate} % klassification af rækkens radius for konvergens.
    \item $R = 0$ i så fald gælder $\lim_{n \rightarrow \infty} T_f(x) = f(x)$ kun i punktet $x = a$,
    hvilket vil sige at taylorrækken slet ikke konvergere ud over punktet $x = a$
    \item $R = \infty$ i så fald gælder $\lim_{n \rightarrow \infty} T_f(x) = f(x)$ for alle $x \in \mathbb{R}$,
    hvilket vil sige at taylorrækken $T_f$ konvergere imod funktionen $f(x)$ for alle værdier.
    \item $R = c$ hvor $c \in \mathbb{R}\setminus\{0\}$ i så fald konvergere taylorrækken kun i området $a - R < X < a + R$.
\end{enumerate}
For at finde konvergens radiusen for en taylorrække kan koeficient kriteriet benyttes, ligesom det kan ved andre potens rækker.
\begin{defn} % Definitation af koeficient kriteriet
    For en potens række er radiusen for konvergens givet ved:
    \[
        \frac{1}{R} = \lim_{n \rightarrow \infty} \left\lvert \frac{c_{n + 1}}{c_n} \right\lvert
    \]
    Hvor $c_n$ for taylorrækker er givet ved $c_n = \dfrac{f^n(a)}{n!}$.
\end{defn} % TODO: BEVIS 


% dette gælder dog ikke for alle funktioner, 
% et simpelt eksempel kunne fx være funktionen $ln(x)$ omkring punktet $a=1$ 
% hvor Taylorpolynomiet kun convergere imod $ln(x)$ hvor $x \in \mathbb{R} \setminus\{(0, 2)\} \land x > 0$. 
% Selv om man lader $n \rightarrow \infty$, hvis $x > 2$ gælder $P_n(x) \not \approx ln(x)$ dog stadig ikke.
% Derudover bør det nævnes at hvis $n \rightarrow \infty$ gælder, kaldes Taylorpolynomiet for en Taylorrække, mere om dette i et senere afsnit. % TODO: Indsæt refference til afsnitet 

% Link til en artikel om ln(x): https://math.stackexchange.com/questions/409214/why-is-domain-of-convergence-of-taylor-rækkes-of-lnx-about-x-1-is-0-2
% Ratio test for Rækker: https://www.youtube.com/watch?v=av947KCWf2U&ab_channel=KhanAcademy 
\subsection*{\textbf{Eksempel:} Maclaurin rækken for $e^x$ og dens radius for konvergens} 
Der ønskes en Maclaurin række for beregning af funktions værdien for funktionen $f(x) = e^x$.
Da $\frac{d}{dx}e^x = e^x$ kan Maclaurin rækken opskrives som:
\[
T_{e^x}(x) = \sum^\infty_{n = 0} e^a \frac{(x-a)^n}{n!}
\]
Da der er tale om en Maclaurin række og $a = 0$ derfor må gælde kan rækken forsimples yderligere:
\[
T_{e^x}(x) = \sum^\infty_{n = 0} \frac{x^n}{n!}  
\]
Herefter beregnes rækkens radius for konvergens:
\[
    \frac{1}{R} = \lim_{n \rightarrow \infty} \left\lvert \frac{(n + 1)!^{-1}f^{(n+1)}(a)}{n!^{-1}f^n(a)} \right\lvert
    = \lim_{n \rightarrow \infty} \left\lvert \frac{\frac{1}{(n + 1)!}e^a}{\frac{1}{n!}e^a} \right\lvert 
    = \lim_{n \rightarrow \infty} \left\lvert \frac{\frac{1}{(n + 1)!}}{\frac{1}{n!}} \right\lvert
\]
Der ses bort fra den absolutte værdi fordi $(n + 1)! > n! > 0$:
\[
    \frac{1}{R}= \lim_{n \rightarrow \infty} \frac{n!}{(n + 1)!} 
    = \lim_{n \rightarrow \infty} \frac{n \cdot (n-1) \cdot \ldots \cdot 2 \cdot 1}{(n + 1) \cdot n \cdot (n-1) \cdot \ldots \cdot 2 \cdot 1}
    = \lim_{n \rightarrow \infty} \frac{1}{n + 1}
\]
Og da $n \rightarrow \infty$ må følgende gælde: $\frac{1}{R} = \frac{1}{\infty} \Leftrightarrow R = \infty$ rækken konvergere altså for alle $x \in \mathbb{R}$.
Derved kan følgende udtryk opstilles:
\[
e^x = \sum^\infty_{n = 0} \frac{x^n}{n!}  
\]
\subsection*{\textbf{Eksempel:} Taylorrækken rækken for $ln(x)$ omkring $a = 1$ og dens radius for konvergens} % Taylorrække for ln(x)
Taylorrækken for $ln(x)$ er givet ved:
\[
    T_{ln}(x) = ln(a) + \frac{ln'(a)}{1!}(x-a)^1 + \frac{ln''(a)}{2!}(x-a)^2 + \ldots
\]
Hvilket kan opskrives pænere som:
\[
    T_{ln}(x) = ln(a) + \sum^{\infty}_{n = 1} \frac{(-1)^{(n+1)}}{n!} \cdot \frac{(n-1)!}{a^n} (x-a)^{n}
\]
$a = 1$ indsættes:
\[
    T_{ln}(x) = \sum^{\infty}_{n = 1} \frac{(-1)^{(n+1)}}{n!} \cdot \frac{(n-1)!}{1} (x-1)^{n}
    = \sum^{\infty}_{n = 1} \frac{(-1)^{(n+1)}(x-1)^{n}}{n}
\]
Nu beregnes Taylorrækkens radius for konvergens: 
\[
    \frac{1}{R} = \lim_{n \rightarrow \infty} \left\lvert \frac{\frac{(-1)^{(n+2)}}{n+1}}{\frac{(-1)^{(n+1)}}{n}} \right\lvert
\]
Fordi den absolutte værdi tages, kan $-1^{(n+2)}$ og $-1^{(n + 1)}$ slettes fra tæller og nævner,
fordi den ene vil være -1 og den anden vil være 1.
\[
    \frac{1}{R} = \lim_{n \rightarrow \infty} \frac{\frac{1}{n+1}}{\frac{1}{n}} = \lim_{n \rightarrow \infty} \frac{n}{n+1} = 1
\]
$\frac{1}{R} = 1 \Leftrightarrow R = 1$ Må gælde fordi både tæller og nævner går imod uendelig. derved konvergere Taylorrækken
kun i området $a-R < x < a+R \Leftrightarrow 0 < x < 2$ derved kan følgende udtryk opstilles: $ln(x) = T_{ln}(x)$ hvor $ 0 < x < 2$