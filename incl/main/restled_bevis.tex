\subsection{Bevis for formlen for restleddet}
Vi vil nu bevise, at restleddet for et $n$'te ordens Taylorpolynomium er givet ved formlen:
\begin{equation}
	R_n(x)=\frac{f^{(n+1)}(s)}{(n+1)!} \cdot (x-a)^{n+1}, \quad n \in \mathbb{N}_0
\end{equation}

Til at bevise dette, skal jeg bruge følgende lemmaer:
\begin{lem}
\label{generel_mvs}
	\begin{equation*}
		\frac{f(b)-f(a)}{g(b)-g(a)}=\frac{f'(c)}{g'(c)},
	\end{equation*}
	Hvor $c$ er et tal mellem $a$ og $b$, $g'(c) \neq 0$, og $g(b) \neq g(a)$. For bevis, se \cite{calc_et}, side 138-144.
\end{lem}

\begin{lem}
\label{rest_dif}
	Den $k$'te afledte af $R_n(x)$ er givet ved:
	\begin{align*}
		R_n^{(k)}(x)&=f^{(k)}(x)-f^{(k)}(a), \quad k \leq n, \quad k \in \mathbb{N}_0 \\
		\implies R_n^{(k)}(a)&=0
	\end{align*}	
\end{lem}

\begin{lem}
\label{potens_dif}
	\begin{align}
		\frac{d^k}{dx^k}(x-a)^n&=\frac{n!}{(n-k)!}(x-a)^{n-k}, \quad k \leq n, \quad k \in \mathbb{N}_0 \\
		\implies \frac{d^n}{dx^n}(x-a)^n&=n!
	\end{align}
\end{lem}


\begin{proof}

Dette bevis er inspireret af beviset for fejlen ved tangentens ligning (Taylorpolynomium af første orden) i \cite{calc_et} side 272. Vi kigger på følgende udtryk, hvor $x>a$. Når størrelsesforholdet er modsat, ligner beviset.

\begin{equation}
	\frac{R_n(x)}{(x-a)^{n+1}}=\frac{R_n(x)-R_n(a)}{(x-a)^{n+1}-(a-a)^{n+1}}
\end{equation}
	Argumentet for, at vi må trække disse værdier fra i tæller og nævner, uden det ændrer på brøkens værdi, er, at $R_n(a)=(a-a)^{n+1}=0$ (jf. lemma \ref{rest_dif}) \\
Nu benyttes den generelle middelværdisætningen (lemma \ref{generel_mvs}) på $R_n(t)$ og $(t-a)^{n+1}$.
\begin{equation}
	\frac{R_n(x)}{(x-a)^{n+1}}=\frac{R_n^{(1)}(x_1)}{(n+1)(x_1-a)^{n}}
\end{equation}
$x_1$ kan være infinitesimalt større end x, og der kan være uendeligt mange værdier af $t$ mellem x og a. Funktionerne i tæller og nævner kan begge differentieres n gange, hvilket for funktionen $(x-a)^{(n+1)}$ kan udledes ved hjælp af lemma \ref{potens_dif}. Derudover kan man altid trække $R_n^{(k)}(a)=0, k \leq n$ (se lemma \ref{rest_dif}) fra i tælleren, og man kan også altid trække $((x-a)^{n+1})^{(k)}$
\begin{equation}
	\frac{R_n(x)}{(x-a)^{n+1}}=\frac{R_n^{(n)}(x_n)}{(n+1)!(x_n-a)}
\end{equation}

\end{proof}