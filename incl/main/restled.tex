\chapter{Restled}

Når man benytter sig af Taylorpolynomier, er det for at approksimere en funktion, og som det ligger i ordet "approksimation", er funktionerne ikke identiske, og der må derfor være en afvigelse fra den oprindelige funktion. Denne afvigelse, også kaldet restleddet eller fejlen, vil vi beskrive i dette afsnit.

\section{Restleddets grundlæggende principper}
\begin{defn}
	I et $n$'tegradspolynomium, $P_n(x)$, kan restleddet, $R_n(x)$, beskrives som 			forskellem mellem funktionen, $f(x)$ som $P_n(x)$ approksimerer, og 		$P_n(x)$.\
	\begin{equation*}
		R_n(x)=|f(x)-P_n(x)|
	\end{equation*}
\end{defn}
Restleddet kan også beskrives på følgende møde:
\begin{equation}
	R_n(x)=\frac{f^{(n+1)}(u)}{(n+1)!}(x-a)^{n+1},
\end{equation}
hvor u er et tal mellem x og a. %plagiat?
\
Det er ikke altid, at man kender værdien af u, så man kan altså ikke nødvendigvis finde ud af præcis hvor stor afvigelsen er (så kunne man jo finde en funktion tilsvarende $f(x)$, som  havde præcis de samme funktionsværdier for alle værdier af $x$ i definitionsmængden).\
Man kan dog ofte finde en nedre og øvre grænse for afvigelsen. Det vil sige, at man ofte kan finde en værdi af $\psi$, der opfylder
\begin{equation}
	|R_n(x)|\leq \psi
\end{equation}
Det vil sige, at man ved, at fejlen ikke er større end $\psi$, og dermed kan man fastlægge et interval $[P_n(x)-\psi,P_n(x)+\psi]$, hvori $f(x)$ er indeholdt. I næste afsnit vil et par eksempler på, hvordan dette kan se ud for nogle bestemte funktioner, figurere.

\section{Eksempler på anvendelse af restled}
