\chapter{Introduktion til Taylor Serier}
Hvis taylor polynomiet indeholder udendeligt mange led, 
kalder man det for en taylor serie i stedet for et polynomie. 
\begin{defn}
    En Taylor serie for den kontinuere funktion $f(x)$ omkring værdien $a$ er givet ved den uendelige sum:
    \[
    \sum^{\infty}_{i=0} \frac{d^i f(a)}{dx^i} \frac{(x-a)^{i}}{i!} = f(a) + \frac{df(a)}{dx} \frac{(x-a)^{1}}{1!} + \frac{d^{2}f(a)}{dx^{2}} \frac{(x-a)^{2}}{2!} + \ldots + \frac{d^{n} f(a)}{dx^{n}} \frac{(x-a)^n}{n!}
    \]
\end{defn}
Idéen med at hældningen af taylor serien og hældningen for funktionen er den samme i værdien $a$ ses også her.
Her er dog uendeligt mange led, hvilket skaber en bedre approksimation omkring værdien $a$. 
En vigtig egenskab for hver taylor serie er dens radius for konvergens % TODO: Beskriv radius for konvergens
\label{def:taylorSerie}
\subsection*{Eksempler på taylor serier} 

% Taylor serie for e^x TODO: beregn radius for konvergens
Der ønskes en serie for beregning af funktions værdien for funktionen $f(x) = e^x$ omkring $a = 0$.
Da $\frac{d}{dx}(e^x) = e^x$ kan taylor serien opskrives som:
\[
\sum^\infty_{i = 0} e^a \frac{(x-a)^i}{i!}
\]
og da $a = 0$ kan serien forsimples yderligere:
\[
\sum^\infty_{i = 0} \frac{x^i}{i!}  
\]
Hvis det antages at seriens radius for konvergens er $\infty$ kan følgende udtryk opstilles for at beregne $e^x$ for $x \in \mathbb{R}$
% TODO: Kan man antage at radiusen for konvergens er $\infty$
\[
e^x = \sum^\infty_{i = 0} \frac{x^i}{i!}  
\]
\subsection*{Konvergens radius for Taylor serier} % Taylor serie for ln(x)
Det er dog ikke altid at taylor serien convergere imod $f(x)$ for alle $x \in D(f(x))$ når $n \rightarrow \infty$,
et eksempel $ln(x)$ blev bl.a. nævnt i kapitel \ref{ch:ItTP} 


% Link til en artikel om ln(x): https://math.stackexchange.com/questions/409214/why-is-domain-of-convergence-of-taylor-series-of-lnx-about-x-1-is-0-2

% Taylor Serier